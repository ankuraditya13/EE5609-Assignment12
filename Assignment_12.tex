\documentclass[journal,12pt,twocolumn]{IEEEtran}

\usepackage{setspace}
\usepackage{gensymb}

\singlespacing


\usepackage[cmex10]{amsmath}

\usepackage{amsthm}

\usepackage{mathrsfs}
\usepackage{txfonts}
\usepackage{stfloats}
\usepackage{bm}
\usepackage{cite}
\usepackage{cases}
\usepackage{subfig}

\usepackage{longtable}
\usepackage{multirow}

\usepackage{enumitem}
\usepackage{mathtools}
\usepackage{steinmetz}
\usepackage{tikz}
\usepackage{circuitikz}
\usepackage{verbatim}
\usepackage{tfrupee}
\usepackage[breaklinks=true]{hyperref}
\usepackage{graphicx}
\usepackage{tkz-euclide}

\usetikzlibrary{calc,math}
\usepackage{listings}
    \usepackage{color}                                            %%
    \usepackage{array}                                            %%
    \usepackage{longtable}                                        %%
    \usepackage{calc}                                             %%
    \usepackage{multirow}                                         %%
    \usepackage{hhline}                                           %%
    \usepackage{ifthen}                                           %%
    \usepackage{lscape}     
\usepackage{multicol}
\usepackage{chngcntr}

\DeclareMathOperator*{\Res}{Res}

\renewcommand\thesection{\arabic{section}}
\renewcommand\thesubsection{\thesection.\arabic{subsection}}
\renewcommand\thesubsubsection{\thesubsection.\arabic{subsubsection}}

\renewcommand\thesectiondis{\arabic{section}}
\renewcommand\thesubsectiondis{\thesectiondis.\arabic{subsection}}
\renewcommand\thesubsubsectiondis{\thesubsectiondis.\arabic{subsubsection}}


\hyphenation{op-tical net-works semi-conduc-tor}
\def\inputGnumericTable{}                                 %%

\lstset{
%language=C,
frame=single, 
breaklines=true,
columns=fullflexible
}
\begin{document}


\newtheorem{theorem}{Theorem}[section]
\newtheorem{problem}{Problem}
\newtheorem{proposition}{Proposition}[section]
\newtheorem{lemma}{Lemma}[section]
\newtheorem{corollary}[theorem]{Corollary}
\newtheorem{example}{Example}[section]
\newtheorem{definition}[problem]{Definition}

\newcommand{\BEQA}{\begin{eqnarray}}
\newcommand{\EEQA}{\end{eqnarray}}
\newcommand{\define}{\stackrel{\triangle}{=}}
\bibliographystyle{IEEEtran}
\providecommand{\mbf}{\mathbf}
\providecommand{\pr}[1]{\ensuremath{\Pr\left(#1\right)}}
\providecommand{\qfunc}[1]{\ensuremath{Q\left(#1\right)}}
\providecommand{\sbrak}[1]{\ensuremath{{}\left[#1\right]}}
\providecommand{\lsbrak}[1]{\ensuremath{{}\left[#1\right.}}
\providecommand{\rsbrak}[1]{\ensuremath{{}\left.#1\right]}}
\providecommand{\brak}[1]{\ensuremath{\left(#1\right)}}
\providecommand{\lbrak}[1]{\ensuremath{\left(#1\right.}}
\providecommand{\rbrak}[1]{\ensuremath{\left.#1\right)}}
\providecommand{\cbrak}[1]{\ensuremath{\left\{#1\right\}}}
\providecommand{\lcbrak}[1]{\ensuremath{\left\{#1\right.}}
\providecommand{\rcbrak}[1]{\ensuremath{\left.#1\right\}}}
\theoremstyle{remark}
\newtheorem{rem}{Remark}
\newcommand{\sgn}{\mathop{\mathrm{sgn}}}
\providecommand{\abs}[1]{\left\vert#1\right\vert}
\providecommand{\res}[1]{\Res\displaylimits_{#1}} 
\providecommand{\norm}[1]{\left\lVert#1\right\rVert}
%\providecommand{\norm}[1]{\lVert#1\rVert}
\providecommand{\mtx}[1]{\mathbf{#1}}
\providecommand{\mean}[1]{E\left[ #1 \right]}
\providecommand{\fourier}{\overset{\mathcal{F}}{ \rightleftharpoons}}
%\providecommand{\hilbert}{\overset{\mathcal{H}}{ \rightleftharpoons}}
\providecommand{\system}{\overset{\mathcal{H}}{ \longleftrightarrow}}
	%\newcommand{\solution}[2]{\textbf{Solution:}{#1}}
\newcommand{\solution}{\noindent \textbf{Solution: }}
\newcommand{\cosec}{\,\text{cosec}\,}
\providecommand{\dec}[2]{\ensuremath{\overset{#1}{\underset{#2}{\gtrless}}}}
\newcommand{\myvec}[1]{\ensuremath{\begin{pmatrix}#1\end{pmatrix}}}
\newcommand{\mydet}[1]{\ensuremath{\begin{vmatrix}#1\end{vmatrix}}}
\numberwithin{equation}{subsection}
\makeatletter
\@addtoreset{figure}{problem}
\makeatother
\let\StandardTheFigure\thefigure
\let\vec\mathbf
\renewcommand{\thefigure}{\theproblem}
\def\putbox#1#2#3{\makebox[0in][l]{\makebox[#1][l]{}\raisebox{\baselineskip}[0in][0in]{\raisebox{#2}[0in][0in]{#3}}}}
     \def\rightbox#1{\makebox[0in][r]{#1}}
     \def\centbox#1{\makebox[0in]{#1}}
     \def\topbox#1{\raisebox{-\baselineskip}[0in][0in]{#1}}
     \def\midbox#1{\raisebox{-0.5\baselineskip}[0in][0in]{#1}}
\vspace{3cm}
\title{Assignment-12}
\author{Ankur Aditya - EE20RESCH11010}
\maketitle
\newpage
\bigskip
\renewcommand{\thefigure}{\theenumi}
\renewcommand{\thetable}{\theenumi}

\begin{abstract}
This document contains the problem related to basis and dimensions.(Hoffman:- Page-49,Q-12 ) 
\end{abstract}
Download the latex-file codes from 
\begin{lstlisting}
https://github.com/ankuraditya13/EE5609-Assignment12
\end{lstlisting}

\section{Problem}
Prove that the space of all $\vec{m} x \vec{n}$ matrices over the field $\vec{F}$ has dimension mn, by exhibiting a basis for this space. 
\section{Solution}
Let $\vec{M}$ be the space of all $\vec{m}\times\vec{n}$ matrices. Let, $\vec{M}_{ij} \in \vec{M}$ be,
\begin{align}
\vec{M}_{ij} = \begin{cases} 0 &  m\neq i, n\neq j \\ 1 & m=i,n=j \end{cases}
\end{align}
For example,
\begin{align}
\vec{M}_{12} = \myvec{0&1&0&\cdots&0\\0&0&0&\cdots&0\\\vdots&\vdots&\vdots&\vdots&\vdots\\0&0&0&\cdots&0}_{mxn}\\
\end{align}
Let $\vec{A} \in \vec{M}$ given as,
\begin{align}
\vec{A} = \myvec{a_{11}&a_{12}&a_{13}&\cdots&a_{1n}\\a_{21}&a_{22}&a_{23}&\cdots&a_{2n}\\\vdots&\vdots&\vdots&\vdots&\vdots\\a_{m1}&a_{m2}&a_{m3}&\cdots&a_{mn}}_{mxn}
\end{align}
Now clearly,
\begin{align}
\vec{a}_{11} = \myvec{a_{11}&a_{12}&\cdots&a_{1n}\\a_{21}&a_{22}&\cdots&a_{2n}\\\vdots&\vdots&\vdots&\vdots\\a_{m1}&a_{m2}&\cdots&a_{mn}} \myvec{1&0&\cdots&0\\0&0&\cdots&0\\\vdots&\vdots&\vdots&\vdots\\0&0&\cdots&0}
\end{align}
\begin{align}
\implies \vec{a}_{11} = \vec{A}\vec{M}_{11}
\end{align}
\begin{align}
\therefore\vec{A} = \sum_{i=1}^{m} \sum_{j=1}^{n} a_{ij}M_{ij} \label{span}
\end{align}
$\implies \vec{M}_{ij}$ span $\vec{M}$. 
Also from equation \eqref{span}, $\vec{A}$=  0 if and only if all elements are zero, that is, 
\begin{align}
\vec{A} = \myvec{a_{11}&a_{12}&\cdots&a_{1n}\\a_{21}&a_{22}&\cdots&a_{2n}\\\vdots&\vdots&\vdots&\vdots\\a_{m1}&a_{m2}&\cdots&a_{mn}}= \myvec{0&0&\cdots&0\\0&0&\cdots&0\\\vdots&\vdots&\vdots&\vdots\\0&0&\cdots&0}
\end{align}
\begin{align}
\implies a_{ij} = 0
\end{align}
Hence, $\vec{M}_{ij}$ are linearly independent as well. Hence, $\vec{M}_{ij}$ constitutes a basis for $\vec{M}$. and number of elememts in basis are mn. Hence dimension of space of all mxn matrices $\vec{M}$ is mn. 
\centerline{\textbf{Hence Proved.}}
\end{document}